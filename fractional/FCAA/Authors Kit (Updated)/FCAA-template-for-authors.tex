%%% Template for AUTHOR's DRAFT paper in FCAA, WITHOUT Journal's head  %%%%%%%%%%%%%%%%%%%%%%%%%%%%%%%%%
%%% by V. Kiryakova, updated Nov. 1, 2014
%%% uses "fcaa.cls", or "fcaa-var.cls" as modifications of "amsart.cls" for the FCAA format,
%%% and auxiliary file "fcaa_style.tex" fixing page margins, fontsize, defs for theorems, proofs etc.
%%% Put the files "fcaa.cls", "fcaa-var.cls" and "fcaa_style.tex" in same directory you prepare the paper

  %\documentclass[twoside,reqno,11pt]{fcaa}  %%% or in case of problems, use as below: %
   \documentclass[twoside,reqno,11pt]{fcaa-var} %

 \input fcaa_style

%%%%%%%%%%%%%%%%%%%%
 \usepackage{hyperref} % Editor will use to create hyperlinks %
 %%% but if the author has problems with the above style file,
 %%% then comment the line \usepackage{hyperref} or replace by this below:
 % \usepackage{upref}
%%%%%%%%%%%%%%%%%%%%

% to have 2-digits numbering for equation, use:
 \def\theequation{\arabic{section}.\arabic{equation}}

%%%%%%  First page footnote for Copyright and Springer logo
 \def\themycopyrightfootnote{\vspace*{3pt}
 \copyright \, Year\,  Diogenes  Co., Sofia
   \par  \noindent pp. xxx--xxx, DOI: ......................
   \hfill  \vspace*{-36pt}
   % \mbox{\includegraphics[scale=0.65]{DeGryuter.eps}}
  }
%%%%%%%%%%%%%%%%%%%%%%%%%%%%%%%%%%%%%%%%%%%%%%%%%%%%%%%%%%%%

  \setcounter{page}{1}
  \thispagestyle{empty}

 %%%%%%%%%%%%%%% begin make title %%%%%%%%%%%%%%%%%%%%%%%%%%%%%
 %%% TITLE: texts in [.] is abbreviated (1st line) title for running heads
 %%% Author(s): put in brackets [.] the short author's name

 \title[VISUALIZATION OF THE FRACTIONAL INTEGRAL \dots]{VISUALIZATION OF THE RIEMANN-LIOUVILLE FRACTIONAL INTEGRAL \\ [3pt] IN ``FCAA'' JOURNAL}
 \author[\normalsize V. Kiryakova, S. Author]{\normalsize Virginia Kiryakova $^1$, Second Author $^2$}

 %%% obligatory give the full and abbreviated authors' names %%%
 %%%%%%%%%%%%%%%%%%%%%%%%%%%%%%%%%%%%%%%%%%%%%%%%%%%%%%%%%%%%%%%
                    % THE BEGINNING %
 \begin{document}

 \vbox to 2.5cm { \vfill }

%%% to make empty space of approx. 2.5cm %%%%%%
%%% will be replaced by Editor with the journal's and publoishers logos %%%%%%%%

 \bigskip \medskip

%%%% Abstract %%%%%%%%%%%%%%%%%%%%%%%%%
 \begin{abstract}

Text of the abstract. Text of the abstract. Text of the abstract.
Text of the abstract. Text of the abstract. Text of the abstract.
Text of the abstract. It should give a comprehensive idea about the
paper's subject and the author's results. The Abstract (and the first 2 pages of the paper) will be
available free at DeGryuter's website for the journal.

 \medskip

{\it MSC 2010\/}: Primary 26A33;
                  Secondary 33E12, 34A08, 34K37, 35R11, 60G22, ...

 \smallskip

{\it Key Words and Phrases}: fractional calculus, Mittag-Leffler
type functions, fractional ordinary and partial differential
equations, ...

 \end{abstract}

 \maketitle

%%%%%%% end make title %%%%%%%%%%%%%%%%%%%%%%%%%%%%%%%%%%
 \vspace*{-16pt}

%%%%%%%% begin papers' body %%%%%%%%%%%%%%%%%%%%%%%%%%%%%

%%%%%%%%%%%%%%%%%%%%%%%%%%% Section 1 %%%%%%%%%%%%%
%\section{Introduction}\label{Sec:1}

\section{First section of the paper}\label{sec:1}

\setcounter{section}{1}
\setcounter{equation}{0}\setcounter{theorem}{0}


Text ... (for details, see \cite{GasRah}, \cite{Rosbl}, \cite{Kir},
\cite{Moak}) ...

%%%% example of definition %%%%
 \begin{definition}\label{Def3}
Text of Definition~\ref{Def3}.
 \end{definition}

   \vspace*{-12pt} %%% example of subsection:
 \subsection{Preliminary results}\label{subsec:1.1}

%%%% example of theorem %%%%%%%%%%
 \begin{theorem}\label{Th1}
Text of Theorem~\ref{Th1} ....
 \end{theorem}

 \proof %%%%%%%%%%%%%
 Give here the proof of Theorem~\ref{Th1}. Example for
equation:
\begin{equation}\label{eq1}
ax^2+bx +c =0.
\end{equation}
 As seen by equation \eqref{eq1}, it is ...
 The proof follows from Ref. \cite{Moak}.
 \proofend %%%%%%%%%%

%%%% example of corollary %%%%
 \begin{corollary}\label{Cor2}
Text of Corollary~\ref{Cor2} ....
 \end{corollary}

 \proof
 Here comes the proof of Corollary~\ref{Cor2}.
 \proofend

%%%%%%%%%%%%%%%%%%%%%%%%%%%%%%%%%%%%%%%%%%%%%%%%%%
\section{Second section of the paper}\label{sec:2}

\setcounter{section}{2}
\setcounter{equation}{0}\setcounter{theorem}{0}


 Text ... As seen in Section~\ref{sec:1}, the equation
(\ref{eq1}), $ a\neq 0$, has the solutions
 \begin{equation}\label{eq2}
 x_{1,2}= {\frac {-b \pm \sqrt{b^2-4ac}}{2a}}\,.
 \end{equation}

 %%% example of example %%%%%
 \begin{example}\label{Ex1}
 Let us take in (\ref{eq2}) ... Then, by Theorem~\ref{Th1}, ...
 \end{example}

 \begin{example}\label{Ex2}
 Under same conditions as in Example~\ref{Ex1}, we consider ...
 \end{example}


The figures should be input in the LaTeX file as eps-files, as below:

%%%%%%%%% example for figure %%%%%%%%%%%%%%%%%%%
  \begin{center}
  \includegraphics[scale=0.4]{figure.eps}
 % \hspace*{2cm}
 % \includegraphics[scale=0.7]{figure2.eps}

 \bigskip

  Fig. 2.1: Control loop
  \end{center}
%%%%%%%%%%%%%%%%%%%%%%%%%%%%%%%%%%%%%%%%%%%%%%%

 Often figures include texts or Latin, Greek etc. letters.
 We kindly ask the authors to take care that no texts fonts need to be embedded, by saving first the text as curves.
 Just in case, along with the obligatory eps-files, please send us also some alternative figures' files in  pdf-, jpg-, etc. format.

\section{Cavalieri Integral}
\label{sec:cav_integral}

 \begin{definition}\label{Def3}
Text of Definition~\ref{Def3}.
 \end{definition}

\begin{definition}\label{def:trans}
A continuous real--valued function $a(y)$ is called a translational function with respect to a continuous real--valued function $f(x)$ on the interval $[a,b]$ if 
$\{x\in\mathbb{R}|a\circ f(x) + z = x\}$ is a singleton, for every $z\in[0,b-a]$ and $a(0) = a$.
\end{definition}

\begin{definition}\label{def:h}
The mapping $h : [a, b] \rightarrow [a',b']$, which maps $x_i^1 \in [a, b]$ to $x_i^2 \in [a',b']$, is defined as
$h(x_i^1) =$ $\{x_i^2 \in [a' ,b'] | a\circ f(x_i^2) + [x_i^1 - a] = x_i^2$ , $a = a(0)\}$
\end{definition}

\begin{theorem}
The mapping $h(x)$ is a strictly monotone continious function. The function $h(x)$ is, therefore, invertable.
\end{theorem}

\noindent
The function $h(x)$ is known as the transformation function. 

\begin{definition}\label{def:g}
The mapping $g:[a', b'] \rightarrow [a, b]$, which maps $x_i^2 \in [a' , b']$ to $x_i^1\in [a, b]$,
is defined as $g(x_i^2) = x_i^2 - a \circ f (x_i^2) + a$.
\end{definition}

\noindent
Note that $g = h^{-1}$ per definition. Moreover $g$ is guaranteed to exist, since $h$ is bijective. 

\begin{definition}\label{def:cav_integral}
Define a partition $\mathcal{P}_1$ of $[a,b]$ to be a set of points $x_0^1, x_1^1,\cdots,x_n^1$, 
where $a = x_0^1 \leq x_1^1 \leq \cdots \leq x_n^1 = b$. For each partition 
$\mathcal{P}_1$ of $[a,b]$ write $\Delta x_k^1 = x_k^1-x_{k-1}^1$. 
%Since both boundaries of any integration strip are necessarily
%translations of the translational function $a(y)$, we can apply the transformation
%function $h$ to the partition $\mathcal{P}_1$. 
If the transformation function $h$ is strictly increasing,
the application of $h$ to the partition $\mathcal{P}_1$ induces a new partition $\mathcal{P}_2 = \{x_0^2, x_1^2,\cdots, x_n^2\}$.
Otherwise, if $h$ is strictly decreasing, the application of $h$ induces a reversed partition $\mathcal{P}_2 = \{x_n^2, \cdots, x_1^2,x_0^2\}$. It
can be assumed that $h$ is strictly increasing, without any loss of generality.  Let 
$M_k = \sup \{f (x), h(x_{i-1}^1) = x_{i-1}^2 \leq x \leq x_i^2 = h(x_i^1)\}$, $m_k = \inf \{f (x), h(x_{i-1}^1) = x_{i-1}^2 \leq x \leq x_i^2 = h(x_i^1)\}$, and set 
\begin{equation}
\label{eq:c_up}
\mathcal{C}_U(\mathcal{P}_1,f,h) = \sum_{k=1}^n M_k \Delta x_k, 
\end{equation}
and
\begin{equation}
\label{eq:c_low}
\mathcal{C}_L(\mathcal{P}_1,f,h) = \sum_{k=1}^n m_k \Delta x_k. 
\end{equation}
The sums in Eq.~\eqref{eq:c_up} and Eq.~\eqref{eq:c_low} are respectively called the the upper and lower Cavalieri sums.
If there is a unique number $I$ that satisfies the inequality $\mathcal{C}_L(\mathcal{P}_1,f,h)\leq I \leq \mathcal{C}_U(\mathcal{P}_1,f,h)$ for all 
partitions $\mathcal{P}_1$ of $[a,b]$, then $I$ is called the Cavalieri integral of $f$ from $a(y)$ to $b(y)$ and is denoted by
\begin{equation}
\int_{a(y)}^{b(y)} f(x) dx.
\end{equation}
\end{definition}

\begin{theorem}
The following Cavalieri, Riemann, and Riemann--Stieltjes integrals are equivalent
\begin{equation}
\int_{a(y)}^{b(y)} f(x) dx = \int_a^b f\circ h(x) dx = \int_{a'}^{b'} f(x)dg(x). 
\end{equation}
\end{theorem}
 
 
 
 
 
 
 
%%%%%%%%%%%%%%%%%%%%%%%%%%%%%%%%%%%%%%%%%%%%%%%%%
\section*{Acknowledgements}

 The author thanks his institution for the support, under Grant No ...

%%%%%%%%%% References %%%%%%%%%%%%%%%%%%%%%%%%%%%%%%%%
%%%% arranged in ALPHABETIC ORDER of Authors' Families
%%%% for articles, insert also DOI numbers if available

 \begin{thebibliography}{99}
 \normalsize

%%%% example for a book %%%%%%%%%%%%%

\bibitem{GasRah}
 G. Gasper, M. Rahman,
 \emph{Basic Hypergeometric Series}.
 Cambridge University Press, Cambridge (1990).

%%%% example for article in FCAA journal %%%%%%%%%%%%%%%%%

\bibitem{Kir}
 V. Kiryakova,
 A brief story about the operators of generalized
fractional calculus.
 \emph{Fract. Calc. Appl. Anal.} \textbf{11}, No 2 (2008), 201--218; DOI: ..........

%%%% example for journal's article %%%%%%%%%%%%%%%%%

\bibitem{Moak}
 D.S. Moak,
 The $q$-analogue of the Laguerre polynomials.
\emph{J. Math. Anal. Appl.} \textbf{81}, No 1 (1981), 20--47. % ; doi: ........

%%%% example for a paper in Proceedings %%%%%%%%%%%%%

\bibitem{Rosbl}
 M. Rosenblum,
 Generalized Hermite polynomials and the Bose-like oscillator
 calculus.
 In: \emph{Operator Theory: Advances and Applications},
 Birkh\"auser, Basel (1994), 369--396.
%%%%%%%%%%%%%%%%%%%%%%%%%%%%%%%%%%%%%%%%%%%%%%%%%%%%

\end{thebibliography} %%%%%%%%%%%%%%%%%%%%%%%%%%%%%%%%

%%%%%%%%%% put authors' addresses here, in \it %%%%%%%%

 \bigskip \smallskip

 \it

 \noindent
   %(First) Author's full postal address
$^1$ Institute of Mathematics and Informatics \\
Bulgarian Academy of Sciences \\
"Acad. G. Bontchev" Str., Block 8 \\
Sofia -- 1113, BULGARIA  \\[4pt]
  e-mail: virginia@diogenes.bg
\hfill Received: November 1, 2014 \\[12pt]
  % Second Author's address
$^2$ Dept. of Physics, University of Bologna\\
Via Irnerio 46, I -- 40126 Bologna, ITALY \\[4pt]
  e-mail: .....

\end{document} %%%%%%%%%%%%%%%%%%%%%%%%%%%%%%%%%%%%%
